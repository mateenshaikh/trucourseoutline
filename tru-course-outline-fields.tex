
\newcommand{\instructorname}     {Mateen Shaikh}
\newcommand{\instructorphone}    {(250) 377-6023}
\newcommand{\instructoremail}    {mshaikh@tru.ca}
\newcommand{\instructorofficehrs}{TBA}
\newcommand{\instructorofficerm} {S350}



\newcommand{\coursecode}     {STAT 3060-01}
\newcommand{\coursetitle}    {Applied Regression Analysis}
\newcommand{\coursevectoring}{(3,1,0)}
\newcommand{\term}           {Fall 2023}

\newcommand{\courseprerequisites}{STAT 2000  with a score of C- and one of MATH 1300/2120/2121 with a score of C-}
\newcommand{\coursecorequisites}{None}
\newcommand{\courseexclusions}{None}





\newcommand{\calendardescription}{
	Students are exposed to the concepts of regression analysis with an emphasis on application. Students will learn how to appropriately conduct residual analysis, perform diagnostics, apply transformations, select and check models, and augment regression such as with weighted least squares and nonlinear models. Students may learn additional topics such as inverse, robust, ridge and logistic regression.
}

\newcommand{\coursedescription}{
}

\newcommand{\courseobjectives}{
	\begin{enumerate}
		\item Derive linear models and corresponding parameter estimates through maximum likeli-hood and least squares.
		\item 	Identify key assumptions of the theory and methods, and corresponding limitations of linear model estimation and the linear models themselves.
		\item	Create an appropriate regression model when appropriate for the given data in a variety of applications and settings, including instances where models give poor results.
		\item	Assess quality of a regression model through classical diagnostics, model selection, and tests.
	\end{enumerate}
}

\newcommand{\textsmaterials}{
	\begin{itemize}
		\item Montgomery DC, Peck EA, Vining GG. \emph{Introduction to linear regression analysis (Fifth edition)}. John Wiley \& Sons.
		\item Notes/code/videos will be uploaded to moodle. 
	\end{itemize}
}

\newcommand{\evaluation}{
	Assignments (best ~2 of ~3)		\dotfill	20\%\\
	Test 1 (~Oct 17) 				\dotfill	15\%\\
	Test 2 (~Nov 21) 				\dotfill 	15\%\\
	Final Exam 						\dotfill	40\%\\
	flex* 							\dotfill 	10\%\\
	\hrule~\\
	Total 							\dotfill 	100\%
	
	\hfill{\footnotesize *The flex mark is the best of invigilated mark. I.e. flex=best(test1,test2,exam).}
	
	If there is an academic integrity violation, then all “best” criteria turns into “worst” criteria throughout the entire course’s evaluation.
}


\newcommand{\educationalaids}{
	
	\subsection*{Formula Sheets}
	Students will have a formula sheet for tests and the exam. Further information will be provided closer to the first test.
	
	\subsection*{Calculator}
	A scientific calculator is allowed. Graphing calculators are not permitted on tests or quizzes. Use of mobile/communication devices as calculator is not allowed during evaluations.
	
	\subsection*{Software}
	
	The course will use R software, which is free to download and operate on Mac, Windows, and Linux. The institution provides computer Windows labs
}


\newcommand{\coursetopics}{
	\begin{enumerate}
		\item Introduction (Chapter 1) 
		\begin{itemize}
			\item	Data collection 
			\item	Uses 
		\end{itemize}
		
		\item Simple Linear Regression (Chapter 2) 
		\begin{itemize}
			\item	Scatterplots  
			\item	model 
			\item	least squares estimation 
			\item	confidence intervals, prediction intervals and tests 
		\end{itemize}
		
		\item 	Multiple Regression (Chapter 3-6) 
		\begin{itemize}
			\item	regression in matrix notation 
			\item	analysis of variance 
			\item	added variable plots 
			\item	interpreting parameter estimates 
			\item	generalized and weighted least squares 
			\item	testing for lack of fit 
			\item	general F tests 
			\item	confidence regions 
			\item	residuals, outliers and influence 
			\item	non-constant variance, non-linearity, non-normality 
			\item	transformations 
		\end{itemize}
		
		\item	Model Building (Chapters 7-10)   
		\begin{itemize}			
			\item  polynomial regression 
			\item	indicator (dummy) variables   
			\item  multicollinearity 
			\item	selecting subsets of variables, selecting models
			\item	AIC, BIC, stepwise, ridge
		\end{itemize}
		
		\item	Validation (Chapter 11) 
		
		\item 	Generalizations of Linear Regression (Chapters 12, 13) 
		\begin{itemize}
			\item	non-linear regression 
			\item	logistic regression 
			\item	poisson regression 
		\end{itemize}
		
		\item	Tree Regression 
	\end{enumerate}
	
}