\documentclass[10pt]{trumathoutline}




\begin{document}
	
	~
	
	\vspace{-1em}
	
	\begin{center}
		\bfseries
		{\Large \underline{Course Outline}}\\[0.5em]
		
		\coursecode
		
		\coursetitle\ \coursevectoring
		
		\term
		
	\end{center}
\begin{tabular}{ r l}
	\textbf{Instructor:}	& \instructorname 	\\
	\textbf{Office:} 		& \instructorofficerm \\
	\textbf{Office Hours:} 	& \instructorofficehrs
\end{tabular}\hfill
\begin{tabular}{ r l}
	 \textbf{Phone:} & \instructorphone\\
	 \textbf{email:} & \href{mailto:\instructoremail}{\instructoremail}\\
	~
\end{tabular}

\vspace{1em}


Thompson Rivers University is located on the Tk’emlups te Secwepemc territory that is situated in the Southern interior of British Columbia within the unceded traditional lands of the Secwepemc Nation.

\section*{Calendar Description}
\begin{quote}
	\calendardescription
\end{quote}
	
%\section*{course Description}
%\begin{quote}
%	\coursedescription
%\end{quote}


\section*{Educational Objectives}
Upon completion of the course the students will be expected to:
\courseobjectives

\section*{Prequisites}
\courseprerequisites

\section*{Corequisites}
\coursecorequisites

\section*{Exclusions}
\courseexclusions

\section*{Texts/Materials}
\textsmaterials

\section*{Student Evaluation}
\begin{minipage}{\textwidth}
\evaluation
\end{minipage}\\


In the event a student misses an evaluation, a mark of zero will be given unless the student contacts the instructor prior to the evaluation/deadline, informing the instructor of the particular situation. Students are responsible for checking the final examination schedule which shall be posted each semester by the Registrar, and for advising the Registrar of any conflicts within the schedule. Attendance at a scheduled final examination is mandatory, and the responsibility is on the student to seek remedy for a missed final exam. Students can refer to TRU Examination Policy (ED 03-9) for more information.


\section*{Accessibility Services}
All TRU students who require accommodations are encouraged to register with Accessibility Services upon registering with TRU. To determine how test and exam accommodations can be arranged, students are encouraged to contact Accessibility Services early on as it may take a few weeks for the accommodations to be arranged. It is a student's responsibility to contact Accessibility Services and meet with an Accessibility Services Advisor to access services. More information can be found in
\href{https://www.tru.ca/current/academic-supports/as.html}{https://www.tru.ca/current/academic-supports/as.html}.

\section*{Attendance Regulations}
A registered student who does not attend the first two events (e.g., lectures/labs/etc.) of the course and who has not made prior arrangements acceptable to the instructor may, at the discretion of the instructor, be considered to have withdrawn from the course and have his/her course registration deleted. A registered student is expected to attend a minimum of 90\% of lectures and seminars for which he/she is enrolled. In the case of deficient attendance without cause, a student may, on recommendation of the instructor to the instructor’s Dean or Chairperson, be withdrawn from a course. Admission to a lecture or seminar may be refused by the instructor for lateness, class misconduct, or failure to complete required work. 
Academic Integrity Policy

TRU students are required to comply with the standards of academic integrity set out in Student Academic Integrity policy (ED 5-0), available at TRU website. Cheating, academic misconduct, fabrication, and plagiarism could result in failure of course or even suspension from TRU. 

\section*{Prior Learning Assessment and Recognition/Challenges}

Students may receive credit for Prior Learning Assessment and Recognition (PLAR) through a formal process designed by a qualified specialist approved by the Department of Mathematics and Statistics. More information can be obtained from the Office of the Registrar.

\section*{Educational Aids}
\educationalaids

\section*{Math Help Centre}
All students are welcome to consult with a math tutor on a drop-in basis, free of charge, at the Math Help Centre, which is located in Science 201.  More information is available on the following webpage: \href{https://www.tru.ca/science/programs/math/math-help-centre.html}{https://www.tru.ca/science/programs/math/math-help-centre.html}. 

\newpage
\section*{Course Topics}
\coursetopics

\end{document}
